\documentclass[10pt]{article}
\usepackage[utf8]{inputenc}
\usepackage[margin=1.2in]{geometry}
\usepackage{amsmath}
\begin{document}

This is the $3^{rd}$ assignment. My name is [yourname] and my student ID is [yourstudentid]\\
\underline{About this document:}

\begin{enumerate}
\item This document uses the font size 10pt.
\item This document has a 1.2-inch margin all the way around.
\end{enumerate}

$ \mbox{Aside from } 4 \times 4 = 16 \mbox{ , this is my favorite equation:}$


\begin{equation}
\mbox{Area of a triangle} = \frac{b \cdot h}{2}.
\end{equation}


Mathematicians are often interested in this integral:
$$ \int e^{x^{2}} \, dx.$$

\noindent This integral is actually \textit{unsolvable}. Solving a system of linear equation is usually much easier.
\begin{align}
2x + 3y - 6z &= 4\\
2y + 5z &= -2\\
x - y + 4z &= 1.
\end{align}

% \begin{alignat}{4}
% 2&x& \,+\, 3&y& \,-\, 6&z& &= 4\\
% &&  \, \,   2&y& \,+\, 5&z& &= -2\\
% &x& \,-\, &y& \,+\, 4&z& &= 1.
% \end{alignat}



\noindent This system can be written in a matrix form like this:

$$
\begin{bmatrix} 
 2 &  3 & -6 &  4\\
 0 &  2 &  5 & -2\\
 1 & -1 &  4 &  1\\
\end{bmatrix}
.$$

The last column in the previous matrix is special, and we can mark it off using the \textbf{array} environment. All of my columns are still center-aligned.

$$
\begin{bmatrix}
\begin{array}{c c c|c}
2 & 3 & -6 & 4\\
0 & 2 & 5 & -2\\
1 & -1 & 4 & 1
\end{array}
\end{bmatrix}
$$

\vfill
The following formulas are written in two columns. You'll notice that this bumps up against the bottom of this page. How did I do it \footnote{Hint: vfill}? I think the alignment here is beautiful, don't you? I put 1 inch between columns.

\begin{alignat}{2}
A &= \pi r^2 \hspace{1in} B &= \beta^2 \\
C &= 2\pi r \hspace{1in} D &= 4x
\end{alignat}

\newpage
Here is a well-known formula:
$$\sum_{i=1}^{n} i = \frac{n(n + 1)}{2}.$$

This formula is not as well known, but it is easy to verify:

$$\int_{0}^{10} x^2\, dx = \frac{1000}{3}.$$

Finally, here is a simple matrix multiplication calculation:

$$
\begin{bmatrix} 
 1 &  0 \\
 2 &  3
\end{bmatrix}
\begin{bmatrix} 
  0 & 5 \\
 -2 & 2
\end{bmatrix}
=
\begin{bmatrix} 
  0 & 5 \\
 -6 & 16
\end{bmatrix}
.
$$














\end{document}
