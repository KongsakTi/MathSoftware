\documentclass[xcolor=dvipsnames]{beamer}
\usepackage{xcolor}
\usepackage{tikz}
\usepackage{amsmath}
\usefonttheme[onlymath]{serif}
\usetikzlibrary{trees}
\usetikzlibrary{shapes}

\tikzstyle{level 1}=[level distance=3.5cm, sibling distance=4cm, thick]
\tikzstyle{level 2}=[level distance=6cm, sibling distance=2cm]

\tikzstyle{door} = [text width=4em, text centered]
\tikzstyle{goat} = [text width=6em, text centered]
\tikzstyle{car} = [text width=8em, text centered]
\tikzstyle{switch} = [text width=10em, text centered]

\tikzset{
    invisible/.style={opacity=0},
    visible on/.style={alt=#1{}{invisible}},
    alt/.code args={<#1>#2#3}{
      \alt<#1>{\pgfkeysalso{#2}}{\pgfkeysalso{#3}}
    },
}


\usetheme{JuanLesPins}
\useoutertheme{miniframes}
\setbeamertemplate{background canvas}[vertical shading][bottom=white,top=structure.fg!25]
\usecolortheme[named=Violet]{structure}


\title{Monty Hall}
\author{Piyasuda Sutikunkarn}
\institute{Mathematical Software\\ \medskip Mahidol University, International College}
\begin{document}

\begin{frame}
\titlepage

\begin{tikzpicture}[remember picture,overlay]  
  \node [xshift=-2.5cm,yshift=-8cm] at (current page.north east)
    {\includegraphics[scale=0.2]{montyanddoors.jpg}};
\end{tikzpicture}

\end{frame}

\section{What is Monty Hall?}
\begin{frame}
\frametitle{Table of Contents}
\tableofcontents[currentsection]
\end{frame}

\begin{frame}
\frametitle{What is Monty Hall?}

\end{frame}

\section{Probability of winning when using switching strategy}
\subsection{Having 3 doors with 1 car, 2 goats, and open 1 door}
\begin{frame}
\frametitle{Table of Contents}
\tableofcontents[currentsection,currentsubsection]
\end{frame}

\begin{frame}
\frametitle{1 car and 2 goats}
\begin{center}
\onslide<1-2> \includegraphics[scale=0.6]{montyhall_3doors.png}
\end{center}

\onslide<2>
\begin{tikzpicture}[remember picture,overlay]  
  \node [xshift=-8.6cm,yshift=-7.5cm] at (current page.north east)
    {\includegraphics[scale=0.1]{choose.PNG}};
\end{tikzpicture}
\end{frame}

\begin{frame}
\begin{tikzpicture}[grow=right, sloped]
\node[door][visible on=<1-3>] {\includegraphics[scale=0.2]{chosen.png}}
    child[visible on=<1-3>] {
        node[goat] {\includegraphics[scale=0.025]{goat.png}}    
            child[visible on=<3>] {
                node[door, label= center:
                    {\Large $1$}] {}
                edge from parent
                node[above] {\includegraphics[scale=0.3]{2ndcase.PNG}}
                node[below]  {}
            }
        edge from parent 
            node[above]  {$\frac{2}{3}$}
    }
    child[visible on=<1-3>] {
        node[car] {\includegraphics[scale=0.13]{car.jpg}}        
        child[visible on=<2-3>] {
                node[door, label= center:
                    {\Large $0$}] {}
                edge from parent
                node[above] {\includegraphics[scale=0.3]{1stcase.jpg}}
                node[below]  {}
            }
        edge from parent
            node[above]  {$\frac{1}{3}$}
    };
\end{tikzpicture}
\end{frame}


\begin{frame}
\LARGE
$$ P(car|switch) = \dfrac{1}{3} \cdot 0 + \dfrac{2}{3} \cdot 1 = \dfrac{2}{3} $$
\end{frame}

\subsection{Having n+m doors with n cars, m goats, and open 1 door}
\begin{frame}
\frametitle{Table of Contents}
\tableofcontents[currentsection,currentsubsection]
\end{frame}

\begin{frame}
\frametitle{n cars, m goats, open 1 door}
\begin{center}
{\LARGE n+m doors} \\
\includegraphics[scale=0.8]{ndoors.jpg}
\end{center}
\end{frame}

\begin{frame}
\begin{tikzpicture}[grow=right, sloped]
\node[door]{\includegraphics[scale=0.2]{chosen.png}}
    child {
        node[goat] {\includegraphics[scale=0.025]{goat.png}}
        edge from parent 
            node[above]  {$\frac{m}{n+m}$}
    }
    child {
        node[car] {\includegraphics[scale=0.13]{car.jpg}}
        edge from parent
            node[above]  {$\frac{n}{n+m}$}
    };
\end{tikzpicture}
\end{frame}

\begin{frame}
\frametitle{The door chosen is \textbf{CAR}}
\onslide<1-5>
\begin{tikzpicture}[remember picture,overlay]  
  \node [xshift=-5.2cm,yshift=0cm] at (current page.center)
    {\includegraphics[scale=0.635]{cangoat.PNG}};
\onslide<2-5>
  \node [xshift=-3.4cm,yshift=0cm] at (current page.center)
    {\includegraphics[scale=0.56]{cancar.PNG}};
\onslide<3-5>
  \node [xshift=-0.7cm,yshift=0cm] at (current page.center)
    {\includegraphics[scale=0.61]{goatleft.PNG}};
\onslide<4-5>
  \node [xshift=3.5cm,yshift=0cm] at (current page.center)
    {\includegraphics[scale=0.617]{carleft.PNG}};
\onslide<5>
  \node [xshift=0.01cm,yshift=-1.65cm] at (current page.center)
    {\includegraphics[scale=0.635]{overall.PNG}};
\end{tikzpicture}
\end{frame}

\begin{frame}
\begin{tikzpicture}[grow=right, sloped]
\node[door] {\includegraphics[scale=0.2]{chosen.png}}
    child {
        node[goat] {\includegraphics[scale=0.025]{goat.png}}
        edge from parent 
            node[above]  {$\frac{m}{n+m}$}
    }
    child {
        node[car] {\includegraphics[scale=0.13]{car.jpg}}        
        child {
                node[door, label= center:
                    {\Large $\frac{n-1}{n+m-2}$}] {}
                edge from parent
                node[above] {}
                node[below]  {}
            }
        edge from parent
            node[above]  {$\frac{n}{n+m}$}
    };
\end{tikzpicture}
\end{frame}

\begin{frame}
\frametitle{The door chosen is \textbf{GOAT}}
\begin{tikzpicture}[remember picture,overlay]  
  \node [xshift=-4.8cm,yshift=0cm] at (current page.center)
    {\includegraphics[scale=0.635]{cangoat.PNG}};

  \node [xshift=-3.2cm,yshift=0cm] at (current page.center)
    {\includegraphics[scale=0.635]{cangoat.PNG}};

  \node [xshift=-0.7cm,yshift=0cm] at (current page.center)
    {\includegraphics[scale=0.62]{goatleft_2nd_.PNG}};

  \node [xshift=3.5cm,yshift=0cm] at (current page.center)
    {\includegraphics[scale=0.617]{carleft_2nd_.PNG}};

  \node [xshift=0.2cm,yshift=-1.6cm] at (current page.center)
    {\includegraphics[scale=0.613]{overall.PNG}};
\end{tikzpicture}
\end{frame}

\begin{frame}
\begin{tikzpicture}[grow=right, sloped]
\node[door] {\includegraphics[scale=0.2]{chosen.png}}
    child {
        node[goat] {\includegraphics[scale=0.025]{goat.png}}    
            child {
                node[door, label= center:
                    {\Large $\frac{n}{n+m-2}$}] {}
                edge from parent
                node[above] {}
                node[below]  {}
            }
        edge from parent 
            node[above]  {$\frac{m}{n+m}$}
    }
    child {
        node[car] {\includegraphics[scale=0.13]{car.jpg}}        
        child {
                node[door, label= center:
                    {\Large $\frac{n-1}{n+m-2}$}] {}
                edge from parent
                node[above] {}
                node[below]  {}
            }
        edge from parent
            node[above]  {$\frac{n}{n+m}$}
    };
\end{tikzpicture}
\end{frame}

\begin{frame}
\begin{align*}
\large
P(car|switch) &= \dfrac{n}{n+m} \cdot \dfrac{n-1}{n+m-2} + \dfrac{m}{n+m} \cdot \dfrac{n}{n+m-2}\\ \\
&= \dfrac{n(n+m-1)}{(n+m)(n+m-2)}
\end{align*}
\end{frame}

\subsection{Having n+m doors with n cars, m goats, and open k doors}
\begin{frame}
\frametitle{Table of Contents}
\tableofcontents[currentsection,currentsubsection]
\end{frame}

\begin{frame}
\begin{tikzpicture}[grow=right, sloped]
\node[door]{\includegraphics[scale=0.2]{chosen.png}}
    child {
        node[goat] {\includegraphics[scale=0.025]{goat.png}}
        edge from parent 
            node[above]  {$\frac{m}{n+m}$}
    }
    child {
        node[car] {\includegraphics[scale=0.13]{car.jpg}}
        edge from parent
            node[above]  {$\frac{n}{n+m}$}
    };
\end{tikzpicture}
\end{frame}

\begin{frame}
\frametitle{The door chosen is \textbf{CAR}}
\onslide<1-5>
\begin{tikzpicture}[remember picture,overlay]  
  \node [xshift=-4.8cm,yshift=0cm] at (current page.center)
    {\includegraphics[scale=0.562]{cankgoat.PNG}};
\onslide<2-5>
  \node [xshift=-2.3cm,yshift=0cm] at (current page.center)
    {\includegraphics[scale=0.52]{cancar.PNG}};
\onslide<3-5>
  \node [xshift=0.2cm,yshift=0cm] at (current page.center)
    {\includegraphics[scale=0.588]{goatleft_k_.PNG}};
\onslide<4-5>
  \node [xshift=4.1cm,yshift=0cm] at (current page.center)
    {\includegraphics[scale=0.57]{carleft.PNG}};
\onslide<5>
  \node [xshift=0.012cm,yshift=-1.61cm] at (current page.center)
    {\includegraphics[scale=0.6748]{overall2.PNG}};
\end{tikzpicture}
\end{frame}

\begin{frame}
\begin{tikzpicture}[grow=right, sloped]
\node[door] {\includegraphics[scale=0.2]{chosen.png}}
    child {
        node[goat] {\includegraphics[scale=0.025]{goat.png}}
        edge from parent 
            node[above]  {$\frac{m}{n+m}$}
    }
    child {
        node[car] {\includegraphics[scale=0.13]{car.jpg}}        
        child {
                node[door, label= center:
                    {\Large $\frac{n-1}{n+m-k-1}$}] {}
                edge from parent
                node[above] {}
                node[below]  {}
            }
        edge from parent
            node[above]  {$\frac{n}{n+m}$}
    };
\end{tikzpicture}
\end{frame}

\begin{frame}
\frametitle{The door chosen is \textbf{GOAT}}
\begin{center}
\includegraphics[scale=0.52]{2ndcase_k_.PNG}
\end{center}
\end{frame}

\begin{frame}
\begin{tikzpicture}[grow=right, sloped]
\node[door] {\includegraphics[scale=0.2]{chosen.png}}
    child {
        node[goat] {\includegraphics[scale=0.025]{goat.png}}    
            child {
                node[door, label= center:
                    {\Large $\frac{n}{n+m-k-1}$}] {}
                edge from parent
                node[above] {}
                node[below]  {}
            }
        edge from parent 
            node[above]  {$\frac{m}{n+m}$}
    }
    child {
        node[car] {\includegraphics[scale=0.13]{car.jpg}}        
        child {
                node[door, label= center:
                    {\Large $\frac{n-1}{n+m-k-1}$}] {}
                edge from parent
                node[above] {}
                node[below]  {}
            }
        edge from parent
            node[above]  {$\frac{n}{n+m}$}
    };
\end{tikzpicture}
\end{frame}

\begin{frame}
\begin{align*}
\large
P(car|switch) &= \dfrac{n}{n+m} \cdot \dfrac{n-1}{n+m-k-1} + \dfrac{m}{n+m} \cdot \dfrac{n}{n+m-k-1}\\ \\
&= \dfrac{n(n+m-1)}{(n+m)(n+m-k-1)}
\end{align*}
\end{frame}

\subsection{Offer an option to switch the door with probability P}
\begin{frame}
\frametitle{Table of Contents}
\tableofcontents[currentsection,currentsubsection]
\end{frame}

\begin{frame}
\begin{tikzpicture}[grow=right, sloped]
\node[door] {\includegraphics[scale=0.2]{chosen.png}}
    child {
        node[goat] {NOT \\ Switch}    
            child[visible on = <3>] {
                node[switch, label= center:
                    {\Large $\frac{n}{n+m}$}] {}
                edge from parent
                node[above] {}
                node[below]  {}
            }
        edge from parent 
            node[above]  {1-P}
    }
    child {
        node[goat] {Switch}        
        child[visible on = <2->] {
                node[switch, label= center:
                    {\large $\frac{n(n+m-1)}{(n+m)(n+m-k-1)}$}] {}
                edge from parent
                node[above] {}
                node[below]  {}
            }
        edge from parent
            node[above]  {P}
    };
\end{tikzpicture}
\end{frame}

\begin{frame}
\begin{align*}
\large
P(car|switch) &= P \cdot \dfrac{n(n+m-1)}{(n+m)(n+m-k-1)} + (1-P) \cdot \dfrac{n}{n+m}\\ \\
&= \dfrac{n(n+m-k-1+Pk)}{(n+m)(n+m-k-1)}
\end{align*}
\end{frame}

\end{document}
