\documentclass{article}
\usepackage[margin=1in]{geometry}
\usepackage{amsmath,amsthm}

\theoremstyle{plain}
\newtheorem{theorem}{Theorem}
\newtheorem{corollary}{Corollary}

\theoremstyle{definition}
\newtheorem{definition}{Definition}
\newtheorem*{note}{Note}

\theoremstyle{remark}
\newtheorem*{notation}{Notation}

\pagestyle{plain}
\begin{document}

\begin{theorem}
So this is what a theorem looks like.
\end{theorem}

\begin{corollary}
A corollary usually comes after the theorem.
\end{corollary}

\begin{definition}
A \textit{matrix} is a set of numbers or symbols placed into a rectangular grid.
\end{definition}

\begin{note}
This note should have no number attached to it.
\end{note}

\begin{notation}
My notation goes here.
\end{notation}


\end{document}
