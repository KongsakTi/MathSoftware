\documentclass{article}
\usepackage[margin=1in]{geometry}
\usepackage{amsmath,amsthm}

\theoremstyle{plain}
\newtheorem{theorem}{Theorem}[section]
\newtheorem{corollary}[theorem]{Corollary}
\newtheorem{lemma}[theorem]{Lemma}
\newtheorem*{main}{Main Theorem}
\newtheorem{proposition}{Proposition}

\theoremstyle{definition}
\newtheorem{definition}{Definition}

\theoremstyle{remark}
\newtheorem*{notation}{Notation}

\pagestyle{plain}
\begin{document}

\begin{main}
This theorem is the point of the whole paper.
\end{main}

\section{The First Section}

\begin{notation}
We should establish our notation before we move on to the Lemma.
\end{notation}

\begin{lemma}
A lemma usually comes before the theorem.
\end{lemma}

\begin{theorem}
This theorem kicks things off.
\end{theorem}

\begin{corollary}
A corollary comes after the theorem, for the most part.
\end{corollary}

\begin{proposition}
This is not numbered consecutively.
\end{proposition}

\section{Another Section}

\begin{theorem}
You can see that the numbering continues here.
\end{theorem}

\begin{corollary}
And it continues consecutively here.
\end{corollary}

\end{document}
