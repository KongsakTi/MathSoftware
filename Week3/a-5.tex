\documentclass[12pt, a4paper]{article}
\usepackage[utf8]{inputenc}
\usepackage[margin=1in]{geometry}
\usepackage{amsthm}

\newtheorem{theorem}{Theorem}

 
 
 
 
 \begin{document}
 	\title{Poincare's $h$-Cobordism and the price of fish}
 	\author{Kongsak Tipakornrojanakit}
 	% \date{}
 	\maketitle
 	
 	\vspace{0.2in}
 	
 	\tableofcontents
 	
 	\vspace{0.4in}
 	
 	
 	\begin{abstract}
 		This paper discusses a new extension of Drivle's Theorem, stated in Theorem 1. For earlier work, see \cite{sw, bjm}.
 	\end{abstract}
 	
 	\section{Drivle's Theorem and the R-O Lemma}
 	In this section, we will state and prove our main result. The fundamental equation of wet fish-pricing is that of Whackabath \cite{sw}:
 	
% 	\equation{f_{xxx} + 3f_{xx} - 2 \cdot  \textnormal{Ker}(f) = 0.}$$
		\begin{equation}\label{hahaha}
		f_{xxx} + 3f_{xx} - 2 \cdot  \textnormal{Ker}(f) = 0.
		\end{equation}
		
 	\noindent We will prove the following:
 	
 	\begin{theorem}
 	[Whackabath's equation]{\normalfont{(\ref{hahaha})} \textit{is hardly ever used.}}
	\end{theorem}
	
 	\section{Gackworth's Lemma in $\Omega$-topologies}
	It is an interesting question whether our Theorem 1 for Whackabath's equation (\ref{hahaha}) (de- fined in Section 1 on page 1) in standard topology can be applied without change in Gackworth's $\Omega$-topologies. A very full discussion of Gackworth's work was given in \cite{bjm}.

 	\begin{thebibliography}{9}
 	\bibitem{bjm}
 	B. J. M. Wilkins,
 	\emph{Topological Dynamics and the Haddock Fishery},
 	Unpublished, 1987.

 	\bibitem{sw}
 	T. I. Strainer \& B.\,J.\,M. Wilkins,
 	\emph{A new result on drivles theorem},
 	Proc. Iceland Cod Fish Soc. Lond. Ser. D \textbf{134} (1993), 8678--8679.
 	\end{thebibliography}
 	
 	
 	\end{document}
 	