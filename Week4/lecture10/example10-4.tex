\documentclass[style=fyma]{powerdot}
\pdsetup{palette=green}
\usepackage{amsthm,amsmath}
\title{Powerdot Example \#2}
\author{Your name}
\date{January 18, 2012}

\theoremstyle{plain}
\newtheorem*{theorem}{Theorem}

\begin{document}

\maketitle

\begin{slide}{An Overview}
\tableofcontents[content=sections]
\end{slide}

\section{Our First Section}

\begin{slide}{The Big Theorem}
  For this example I'm using the \texttt{fyma} style \pause with the \texttt{green} palette. \pause

\vspace{.5in} I skipped 0.5 inches there.
\end{slide}

\begin{slide}{Trying out Overlays}

\onslide{1-2}{We used} \onslide{2}{\texttt{pause} on the previous slide, not here though.  Here we only
use \texttt{onslide}.}

\vspace{.2in}

\onslide{1,3-}{Here is the easiest matrix in the world to type:
\begin{equation}
\begin{pmatrix}
1 & 2 \\ 3 & 4
\end{pmatrix}
\end{equation}}

%Overlays within \texttt{itemize} are also
%\begin{itemize}
%\item<4> easy to understand;
%\item<5-6> easy to replicate; and
%\item<4,6> easy to explain.
%\end{itemize}

\end{slide}

\end{document}
